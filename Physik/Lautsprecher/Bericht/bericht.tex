\documentclass{report}
\usepackage{graphicx}
\usepackage{enumitem}
\usepackage{pdfpages}
\usepackage{multicol}
\usepackage[a4paper, left=1in, right=1in, top=1in, bottom=1in]{geometry}
\usepackage[ngerman]{babel}
\usepackage{fancyhdr}
\pagestyle{fancy}

% clear Footer
\fancyfoot{}


% Header
\fancyhead[L]{Andrin Tim Lerjen, \\ Nadja Rahm, Niklas Fister}
\fancyhead[C]{Kantonnschule Wettingen - Physik}
\fancyhead[R]{\thepage}
\renewcommand{\headrulewidth}{1pt}

% Title
\title{\Huge\textbf{Bericht Physik - Lautsprecher}}
\author{Andrin Tim Lerjen, Nadja Rahm, Niklas Fister}
\date{\today}

% makeing title
\begin{document}
\pagenumbering{Roman}
\maketitle

% Abstract
\newpage
\begin{abstract}
    In dem Folgenden Bericht geht es um die Bau eines Lautsprechers aus alltäglichen Materialien.

    Es wurde ein Leistungsstarken Lautsprecher gebaut, was vor allem an den starken Magneten liegt.

    Die Bauart aus leichten Materialien für den Schallerzeuger und dem stabilen Klangkörper sorgen für ein gutes Klangerlebnis.
    
\end{abstract}

% Table of contents
\newpage
\tableofcontents

% report
\newpage
\pagenumbering{arabic}
\setcounter{page}{1}

% Theory
\chapter{Theorie}
\section{Geschichte des Lautsprechers}
Für einen Lautsprecher ist eine Spule und ein Magnet das Wichtigste. Ohne das ist man gut am A und darf weinen\dots

Herr Perucchi sieht aber gerne Menschen weinen und ist ein toller lieber Sadist\dots
\section{Ideale Bauweise}

\end{document}