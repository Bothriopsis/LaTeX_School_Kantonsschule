\documentclass{article}
\usepackage{graphicx}
\usepackage{enumitem}
\usepackage{pdfpages}
\usepackage[a4paper, left=1in, right=1in, top=1in, bottom=1in]{geometry}
\usepackage[ngerman]{babel}

\title{\huge\textbf{Trümmerliteratur}}
\author{Niklas Fister}
\date{\today}

\begin{document}
\maketitle
\newpage

\section{Buch Texstelle}
\subsection{Gefühle zu der Textstelle}
Es schien sehr düster, jedoch gab einem der Alte Mann halt. Es war ein sehr verwirrender Text und speziell ein zweites Mal zu hören. \\
Es war sehr traurig, einsam und kalt. Es war seltsam und unsicher (die Personen gaben dieses Gefühl). Man bekam ein beängstigendes Gefühl.

\subsection{Inhaltliche Analyse}
\begin{itemize} [parsep=0pt]
    \item Schutt und Asche mit allem Grau zeigen das Traurige und Leiden.
    \item Die krummen Beine des Mannes widerspiegeln die Armut
    \item Das Vertrauen ist Anfangs nicht gegeben. Die Menschen allgemein vertrauten sich nicht mehr.
    \item Die Charaktere sind in positiv und negativ unterteilt:\\
    Den Mut Jürgens ist eine gute Eigenschaft und wiederspiegelt die Bevölkerung. Er hängt am alten Leben fest und kann nicht loslassen. \\
    Der Mann hat die Ressourcen, um das ganze zu bewerkstelligen. Er hilft den Neuanfang zu bewerkstelligen.
    \item \textbf{Höhepunkt der Geschichte} \\
    "Sie esssen doch von den Toten", wo man merkt, dass er seinen Bruder verloren hat.
    \item \textbf{Wendepunkt der Geschichte} \\
    Der Moment, wo der Junge ein Kind haben darf.
    \item \textbf{Zweck} war, zu zeigen, wie schlimm ein Krieg ist und dass es dannach dennoch einen Hoffnungsschimmer gibt. Zudem zeigt es, dass die Unschuldigen auch betroffen sind. Es zeigt auch die Verbundenheit der Personen.
\end{itemize}

\subsection{Fromanalyse}
\begin{itemize}[parsep=0pt]
    \item Die Geschichte ist mehr beschreibend und weniger wertend.
    \item Paradaktisches Satzgefüge $\rightarrow$ aneinandergereite Sätze
    \item Es wurden viele beschreibende Adjective. $\rightarrow$ \textbf{wenig} show don't tell
    \item Die Begegnung schien nicht auf Augenhöhe zu sein.
    \item Der Lehrer wurde als nicht glaubwürdig eingestuft, da Autoritäten und vor allem auch Schulen in der NS-Zeit sehr von den Vorgaben geprägt waren und somit der Glaube schwand.
    \item Es wurde in einer auktorial er-/sie-Erzählung geschildert.
    \item Es ist eine lineare Handlungskette $\rightarrow$ Es gibt fast keine Rückblicke in der Geschichte.
    \item Hasen und Ratten sind Gegensätze. Kanninchen wiederspiegeln Hoffnung, Nahrung und das Gute. Ratten sind wiederspiegeln das Schlechte und die Furcht. Sie stehen für das Elend.
    \item Der allte Mann ist Hoffnungsbringer in der schlechten Zeit. Er bringt die weissen Kaninchen und will damit den Menschen helfen. Er will sie auf manipulitaive Weise aus der Situation bringen; er zeit klar Manipulation.
    \item Der Kontrast zwischen alt und jung wird gezeigt mit dem alten Mann und Jürgen.
    \item Die Atmosphäre verändert sich von Düster zu herzig. Anfgans wird alles als trauriger geschildert und mit den Kaninchen wird ein Samen gesäht für Hoffnung und Schönes.
    \item Vor allem die Symbole dienen als rethorische Stilmittel.
    \item Es gibt keine Interjektionen.
\end{itemize}

\end{document}