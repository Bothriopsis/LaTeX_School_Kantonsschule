\documentclass{article}
\usepackage{graphicx}
\usepackage{enumitem}
\usepackage{pdfpages}
\usepackage{multicol}
\usepackage[a4paper, left=1in, right=1in, top=1in, bottom=1in]{geometry}
\usepackage[ngerman]{babel}
\usepackage{fancyhdr}
\pagestyle{fancy}

% Header
\fancyhead[L]{Niklas Fister}
\fancyhead[C]{Kantonnschule Wettingen - Deutsch}
\fancyhead[R]{\thepage}

% clear Footer
\fancyfoot{}

% Title
\title{\Huge\textbf{Franz Kafka}}
\author{Niklas Fister}
\date{\today}

\begin{document}
\maketitle
\thispagestyle{empty}
\newpage

% Werk Heimkehr Bearbeitung
\part{Heimkehr}
\section{Halt vor der Türe}
\begin{itemize}[parsep=0pt]
    \item Familienkonflikt: "nützt ihnen nichts" $\rightarrow$ hat das Gefühl, nicht willkommen zu sein und Angst davor, den elterlichen Erwartungen nicht zu entsprechen.
    \item Selbstzweifel und Unwissenheit: Entfremdung. Viele Erfahrungen und Erlebnisse haben dazu geführt, dass etwas Vertrautes plötzlich fremd ist.
    \item Angst vor Konfrontation, Vaterkonflikt. will sein Geheimnis wahren (Schluss). Konflikt mit väterlicher Dominanz wird als einschränkend limitierend empfunden.
    \item Veränderung vor Ort. Damaliges noch verhanden? Eltern noch da? Angst vor Veränderung.
    \item Beide Seiten haben sich stark voneinander distanziert (Geheimnis), man ist sich fremd geworden.
    \item Will sich nicht mehr seiner Kindheit konfrontieren. Erkennt mit erwachsenem Blick die Tramata der Kindheit und will sich diesen stellen
    \item Geheimnis: Entwicklung, die man ohne einander gemacht hat. Fremdheit, Distanzierung verhindert, dass man dieses Geheimins nicht mehr teilen möchte. Graben zu gross? Einseitge Entwicklung (Eltern gleich geblieben, Distanzierung als Selbstschutz)? Anst vor elterlicher Enttäuschung.
    \item Gamilie muss den ersten Schritt tun $\rightarrow$ Zweifel und Zögern eine zu grosse Hürde / Positiv, da er erwachsen geworden ist. Ist nun seine eigene Familie...
\end{itemize}

\newpage
\section{Schreibauftrag}
\textbf{Die Türe}
\begin{multicols}{2}
    Aus der Ferne sah ich ihn herantreten. Das lange vergessene Gesicht erkannte ich erstmals nicht. Das letzte Mal, als er durch mich hindurchschritt war, als alles noch gut war. Als die ich nicht die Beiden Seiten trennte.
    
    Langsam schritt er mir entgegen. Er machte halt und schien zu überelegen. Was wohl in ihm vorging konnte ich nicht sagen. Seine züge waren schwer zu deuten.

    Das er wieder zurückgekeht war, wunderte mich nicht. Schon lange erwartete ich ihn wieder. Er war allein zu schwach, kam ohne seine Eltern nicht zurecht und prokastinierte doch den ganzen Tag immer. Deshalb musste man ihn los schicken, etwas selbst zu errreichen.

    Doch war er nun schon wieder zurückgekeht. Vielleicht hielt er an, da er sich schähmte. Sich schähmte, für wer er war, was er tat, was er eben nicht tat.

    Er schien immer noch zu zögern. Schien fast schon Angst vor mir zu haben. Mit kritischem blick schaute er über die Landschaft, welche uns beide trennte. Die alten Utensilien, welche schon lange nicht mehr in Gebrauch waren.

    Eine Stille herrschte. Nur unterbrochen durch das rauschen des Windes, das rascheln der Blätter. Selten ein Pfeifen der Vögel. Sonst volkommene Stille.

    Immernoch stand er da. Schien zu lauschen. Bewegte sich nicht. Wollte er, dass man auf ihn zuging? Das ich mich von selbst vor ihm öffnete? Den Weg zu dem öffnete, was ihn verarchtete?

    Innen merkte man nichts von ihm. Man sass nur am Esstisch und ass. Viel mehr war im hohen Alter nicht mehr möglich.

    Brauchen würden sie sich gegenseitig. Ich war jedoch dazwischen. Trennte die beiden so unterschiedlichen Welten. Ob sie sich vertragen würden, weiss niemand. Vermutungen waren das Einzige.
\end{multicols}
\newpage


\part{Vor dem Gesetz}
\section{Gefühle durch die Parabel}
\begin{itemize}[parsep=0pt]
    \item Es entsteht Neugiere, was wirklich hinter der Türe/ hinter dem Gesetz steckt. Wieso will nur er wissen, was hinter dem Gesetz ist.
    \item Verwirrung wird gestiftet, wieso der alte Mann so stark darauf beharrt. Er redet von mehreren Türhütern.
    \item Frust wird gewekckt, da er sein ganzes Leben lang auf etwas wartete, was nicht geschah. Erinnert an eigenes Leben, wenn man lange auf eine Antwort wartete, aber nie eine erhalten hat.
    \item Man hat das Verlangen nach einem Ende der Geschichte, obschon das Ende sehr definitv ist.
\end{itemize}

\textbf{Wirklichkeitsbezüge}

\begin{itemize}
    \item Psychologische Perspektive
    \item Soziologische Perspektive
    \item Theologische Perspektive
    \item Philosophische Perspektive
    \item Weitere Perspektive
\end{itemize}

\section{Philosophische Perspektive}
Wie viele Räume gibt es? Kann es sein, dass es unendlich viele gibt? Kann er aber auch sein, dass der Türhüter gelogen hat und es nur einen Raum gibt?

Entweder das Gesetz ist banal eine Richtlinie, an welche man sich halten sollte. Man kann deuten, dass es an die Art des korrekten Lebens ist.
Es zeigt seinen inneren Konflikt, die korrekte Lebensart zu finden. Das Gesetz ist das Ziel hierbei und die Türhüter das Hindernis.
Das erste Hindernis ist er selbst. Er schafft es nicht, seine eigene Art und Lebensweise zu ändern und scheitert daran.

Die Türhüter behauptet nur, er sei mächtig und beweist es nicht. Der Mann glaubt es ihm aber und malt es sich selbst dann auch so aus.
Bezogen auf das Leben könnte man denken, es ist etwas wovor man Angst hat und sich darin reinsteigert und es damit verschlimmert.

Ebenfalls kann man das Gesetz als Wissen der absoluten Wahrheit anzusehen und die Türhüter als Limite der menschlichen, kongnitiven Grenzen ansehen. Dem Mann fehlt der Mut bestehende Regeln zu hinterfragen, um Fortschritt zu machen.

Der Glanz am Ende und davor die Flöhe auf dem Türhüter könnten ein letzter Hoffnungsschimmer sein, welcher mit dem Tod dann aber erlischt.

Man kann sein Leben als nutzlos betrachten, da er nicht sein Ziel erreicht hat.

\section{Metareflexive Perspektive}
Parabel thematisiert die Unmöglichkeit der eigenen Deutbarkeit. 

Landmann = Leser, der den Text, dessen Gesetz, verstehen will, also Eintritt in den Text möchte. Seine Vergeblichkeit, in das Gesetz einzutreten, spiegelt hier Vergeblichkeit des Lesers, Kafkas Text zu verstehen. Der „Sinn der Parabel“ wird sich dem Leser stets entziehen, ähnlich wie sich das Gesetz stets dem Mann vom Lande entziehen wird. (Die Darstellung des Wegs als Ziel)

Türhüter: Unauflösbare Widersprüche im Text. Unsere eigene Verkopftheit, unser Deutungswahn (Man will verstehen… Es kann auch einfach emotional von Bedeutung sein, statt eine abschliessende Deutung anzustreben). Rationales Denken.
 
Zudem: "In manchen Fällen dient die Parabel häufig mehr dem Ausdruck von Befindlichkeiten als der Veranschaulichung einer Lehre oder Weisheit." 
 
Wohl auch bei Kafka der Fall. Qualität liegt darin, Gefühle auszulösen und Fragen aufzuwerfen. Eine Lehre oder Weisheit steht kaum im Vordergrund. Vielleicht auch eine Schule, Unheimliches und Ungewisses auszuhalten, sich mit existentiellen Situationen auseinanderzusetzen, die sich einer klaren Lösung entziehen. Dem sind wir ständig ausgesetzt, oft versuchen wir uns abzulenken, das zu verdrängen.



\end{document}