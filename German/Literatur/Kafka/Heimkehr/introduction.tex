\documentclass{article}
\usepackage{graphicx}
\usepackage{enumitem}
\usepackage{pdfpages}
\usepackage{multicol}
\usepackage[a4paper, left=1in, right=1in, top=1in, bottom=1in]{geometry}
\usepackage[ngerman]{babel}
\usepackage{fancyhdr}
\pagestyle{fancy}

% Header
\fancyhead[L]{Niklas Fister}
\fancyhead[C]{Kantonnschule Wettingen - Deutsch}
\fancyhead[R]{\thepage}

% clear Footer
\fancyfoot{}

% Title
\title{\huge\textbf{Franz Kafka - Heimkehr}}
\author{Niklas Fister}
\date{\today}

\begin{document}
\maketitle
\newpage

\section{Halt vor der Türe}
\begin{itemize}[parsep=0pt]
    \item Familienkonflikt: "nützt ihnen nichts" $\rightarrow$ hat das Gefühl, nciht willkommen zu sein und Angst davor, den elterlichen Erwartungen nicht zu entsprechen.
    \item Selbstzweifel und Unwissenheit: Entfremdung. Viele Erfahrungen und Erlbenisse haben dazu geführt, dass etwas Vertrautes plötzlich fremd ist.
    \item Angst vor Konfrontation, Vaterkonflikt. will sein Geheimnis wahren (Schluss). Konflikt mit väterlicher Doninanz wird als einschränkend limiterend empfunden.
    \item Veränderung vor Ort. Damaliges noch verhanden? Eltern noch da? Angst vor Veränderung.
    \item Beide Seiten haben sich stark voneinander distanziert (Geheimnis), man ist sich fremd geworden.
    \item Will sich nicht mehr seiner Kindheit konfrontieren. Erkennt mit erwachsenem Blick die Tramata der Kindheit und will sich diesen stellen
    \item Geheimnis: Entwicklung, die man ohne einander gemacht hat. Fremdheit, Distanzierung verhindert, dass man dieses Geheimins nciht mehr teilen möchte. Graben zu gross? Einseitge Entwicklung (Eltern gleich geblieben, Distanzierung als Selbstschutz)? Anst vor elterlicher Enttäuschung.
    \item Gamilie muss den ersten Schritt tun $\rightarrow$ Zweifel und Zögern eine zu grosse Hürde / Positiv, da er erwachsen geworden ist. Ist nun seine eigene Familie...
\end{itemize}
\end{document}